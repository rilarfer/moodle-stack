%%%%%%%%%%%%%%%%%%%%%%%%%%%%%%%%%%%%%%%%%%%%%%%%%%%
%
% Un sencillo tutorial de sintaxis en papel para estudiantes.
%
% Escrito por primera vez por Jonathan Watkins
% Marzo de 2013, Universidad de Birmingham
% Traducido a español por Ricardo Largaespada
% Octubre 2023, Universidad Nacional de Ingeniería
%
%%%%%%%%%%%%%%%%%%%%%%%%%%%%%%%%%%%%%%%%%%%%%%%%%%%
\documentclass[a4paper]{article}
\usepackage{amsmath}
\usepackage{color}

\usepackage{fullpage}
\usepackage[margin=.5in,landscape]{geometry}
\usepackage{float}
\usepackage{multicol}

\setlength{\parindent}{0in}
\newcommand{\stack}[1]{{\color{red}\tt #1}}
\pagestyle{empty}

\begin{document}

\begin{multicols}{2}
\section*{Tutorial sobre la sintaxis STACK}
Este tutorial está diseñado para proporcionar la información y la sintaxis necesarias para sacar el máximo provecho de STACK. Las preguntas en STACK pueden pedir respuestas de diferentes maneras:
\begin{itemize}
\item Valores Numéricos.
\item Expresiones algebraicas.
\item Matrices.
\end{itemize}
Cada pregunta tendrá una o varias casillas para que introduzca su respuesta.  En algunos casos se incluirá una matriz.

\section*{Números}
Debes escribir los números sin espacios y sin utilizar comas para agrupar los dígitos.  A menos que se le indique lo contrario, utilice siempre fracciones en lugar de decimales (por ejemplo, utilice (1/4) en lugar de 0.25).
\begin{itemize}
\item $\pi$ se introduce ya sea como \stack{pi} o \stack{\%pi}.
\item La base de los logaritmos naturales, ($e \approx 2.718 \dots$) se introduce como \stack{e} o \stack{\%e}.
\item La unidad imaginaria $i=\sqrt{-1}$ se introduce como \stack{i} o \stack{\%i}. También puede utilizar \stack{sqrt(-1)}, o \stack{(-1)$^{\wedge}$(1/2)}, teniendo cuidado con los corchetes.
\item También puedes utilizar la notación científica para los números grandes, e.g. $1000$ puede introducirse como \stack{1E+3}.
     {\em Nota:} hágalo sólo si la pregunta se lo permite, ya que las representaciones de números en coma f\mbox{}lotante no suelen estar permitidas.
\end{itemize}

\section*{Suma, resta, multiplicación y división}
\begin{itemize}
\item Para expresar una suma de dos cantidades, utilice el signo $+$, e.g., \stack{x+y}.
\item Para expresar una resta de una cantidad de otra, utilice el signo $-$, e.g., \stack{x-y}.
\item Utiliza un asterisco para multiplicar. Olvidar esto es, con mucho, la fuente más común de errores de sintaxis. Por ejemplo,
\begin{itemize}
\item Deberías introducir $3x$ como \stack{3*x}.
\item Deberías introducir $x(ax+1)(x-1)$ como \stack{x*(a*x+1)*(x-1)}.
\end{itemize}
\item STACK a veces intenta insertar estrellas por usted donde no hay ambigüedad, p.ej.~\stack{2x} o \stack{(x+1)(x-1)}. Esta suposición no puede ser perfecta, ya que la notación matemática tradicional a veces es ambigua. Compara $f(x+1)$ y $x(t+1)$. Por lo tanto, es una buena práctica utilizar siempre estrellas para la multiplicación.
\item Asegúrate de utilizar el orden de precedencia correcto. El acrónimo PEMDAS (Paréntesis, Exponentes, Multiplicación, División, Adición, Sustracción) es un recordatorio útil del orden normal.
\end{itemize}

\section*{Potencias}
Utilice un signo de intercalación ($^{\wedge}$) para elevar algo a una potencia: por ejemplo, $x^2$ debe introducirse como \stack{x$^{\wedge}$2}. Puedes obtener un signo de intercalación manteniendo pulsada la tecla MAYÚS y pulsando la tecla 6 en la mayoría de los teclados. Las potencias negativas o fraccionarias necesitan paréntesis:
\begin{itemize}
\item $x^{-2}$ debe introducirse como \stack{x$^{\wedge}$(-2)}.
\item $x^{\frac{1}{3}}$ debe introducirse como \stack{x$^{\wedge}$(1/3)}.
\item Tenga cuidado con los números negativos, e.g. \stack{(-4)$^{\wedge}$2}.
\end{itemize}

\section*{Paréntesis}
Los paréntesis son importantes para agrupar términos en una expresión. Esto es especialmente cierto en STACK, ya que utilizamos una entrada unidimensional en lugar de las matemáticas escritas tradicionales. Intente desarrollar conscientemente un sentido de cuándo necesita corchetes y evite poner demasiados. Por ejemplo,
\begin{itemize}
\item $\frac{a+b}{c+d}$ debería introducirse como \stack{(a+b)/(c+d)}.
\item Si escribes \stack{a+b/(c+d)}, entonces STACK pensará que te refieres a $a + \frac{b}{c+d}$.
\item Si escribes \stack{(a+b)/c+d}, entonces STACK pensará que te refieres a $\frac{a+b}{c} + d$.
\item Si escribes \stack{a+b/c+d}, entonces STACK pensará que te refieres a $a + \frac{b}{c} + d$.
\end{itemize}

\section*{Matrices}
Puede introducir cualquier matriz de tamaño $m\times n$.
\begin{itemize}
\item Primero debe inicializar una matriz escribiendo \stack{matrix()}. Lo que esté dentro de los corchetes especificará toda la información de la matriz. Cada fila se expresa dentro de un corchete \stack{[ ]}, con cada término separado por una coma. Cada fila debe estar separada por una coma, por ejemplo, ~\stack{[ ],[ ]}. Esto especificará completamente una matriz en Maxima.
\item \stack{matrix([a,b,c],[d,e,f],[g,h,i])} le dará la matriz que se muestra a continuación.
\begin{displaymath}
\left( \begin{array}{ccc} a & b & c \\ d & e & f \\ g & h & i \\ \end{array} \right)
\end{displaymath}
\end{itemize}
\section*{Funciones}
Funciones estándar: Las funciones, como $\sin$, $\cos$, $\tan$, $\exp$, $\log$, etc., pueden introducirse utilizando sus nombres habituales. Sin embargo, el argumento debe ir siempre entre paréntesis: $\sin{(x^2 + 1)}$ debe introducirse como \stack{sin(x$^{\wedge}$2 + 1)}, $\ln{(3+y)}$ debe introducirse como \stack{ln(3+y)}, etc.

\section*{Logaritmos}
Puede utilizar \stack{log(x)} para el logaritmo natural de $x$. Tenga en cuenta que esto comienza con una l minúscula, no una L mayúscula. Actualmente en STACK tanto \stack{ln} como \stack{log} son los logaritmos naturales con base ($e \approx 2.718 \dots$).

\section*{Función Exponencial}
Siempre se debe escribir \stack{exp(x)} para $e^x$. (Escribir \stack{e$^{\wedge}$x} debería funcionar en STACK, pero te lleva a malos hábitos cuando intentas otras formas de programación más tarde).

\section*{Función valor absoluto}
La función módulo, a veces llamada el {\em valor absoluto de $x$}, se escribe como $|x|$ en notación tradicional. Debe introducirse como \stack{abs(x)}.

\section*{Funciones Trigonométricas y funciones Hiperbólicas}
STACK utiliza radianes para los ángulos, ¡no grados!
\begin{itemize}
\item La función \(\frac{1}{\sin(x)}\) debe escribirse como \stack{csc(x)} en lugar de cosec(x) (o puedes llamarlo \stack{1/sin(x)} si lo prefieres).
\item $\sin^2{x}$ debe introducirse como \stack{(sin(x))$^{\wedge}$2} (que es lo que realmente significa, después de todo). Similarmente para $\tan^2{x}$, $\sinh^2{x}$) etc.
\item Recuerde que $\sin^{-1}{x}$ tradicionalmente significa $t$ tal que $\sin(t) = x$, que, por supuesto, es completamente diferente del número $(\sin{x})^{-1} = \frac{1}{\sin{x}}$. Esta notación tradicional es realmente bastante desafortunada y no es utilizada por STACK; en su lugar, $\sin^{-1}{x}$ debe introducirse como \stack{asin(x)}. Del mismo modo, $\tan^{-1}{x}$ debe introducirse como \stack{atan(x)} y así sucesivamente. La inversa $\sin$, asin (o arcsin en su totalidad), es una forma mucho mejor de escribir funciones trigonométricas inversas, ¡ya que elimina cualquier duda sobre lo que uno está tratando de expresar!
\end{itemize}

%\newpage

\section*{Símbolos}
Los símbolos se utilizan a menudo en matemáticas y física, por lo que es posible que deba utilizarlos en sus respuestas. En la mayoría de los casos, será necesario utilizar letras griegas. Maxima le pide que escriba literalmente la letra, sin ningún antecedente. Por ejemplo, la letra $\omega$ puede introducirse escribiendo \stack{omega} y la letra $\Omega$ puede introducirse escribiendo \stack{Omega}. Si desea la versión en mayúscula de una letra, escriba en mayúscula la primera letra de su grafía. Para su comodidad, se ha proporcionado una lista completa de las letras griegas y sus grafías.
\begin{table}[H]
\begin{minipage}{0.22\linewidth}
\centering
\begin{tabular}{|c|c|c|} \hline
A & $\alpha$ & Alpha\\ \hline
B & $\beta$ & Beta\\ \hline
$\Gamma$ & $\gamma$ & Gamma \\ \hline
$\Delta$ & $\delta$ & Delta \\ \hline
E & $\epsilon$ & Epsilon \\ \hline
Z & $\zeta$ & Zeta \\ \hline
\end{tabular}
\end{minipage}
\hspace{0.2cm}
\begin{minipage}{0.22\linewidth}
\centering
\begin{tabular}{|c|c|c|} \hline
H & $\eta$ & Eta \\ \hline
$\Theta$ & $\theta$ & Theta \\ \hline
I & $\iota$ & Iota \\ \hline
K & $\kappa$ & Kappa \\ \hline
$\Lambda$ & $\lambda$ & Lambda \\ \hline
M & $\mu$ & Mu \\ \hline
\end{tabular}
\end{minipage}
\hspace{0.2cm}
\begin{minipage}{0.22\linewidth}
\centering
\begin{tabular}{|c|c|c|} \hline
N & $\nu$ & Nu \\ \hline
$\Xi$ & $\xi$ & Xi \\ \hline
O & o & Omicron \\ \hline
$\Pi$ & $\pi$ & Pi \\ \hline
P & $\rho$ & Rho \\ \hline
$\Sigma$ & $\sigma$ & Sigma \\ \hline
\end{tabular}
\end{minipage}
\hspace{0.2cm}
\begin{minipage}{0.22\linewidth}
\centering
\begin{tabular}{|c|c|c|} \hline
T & $\tau$ & Tau \\ \hline
$\Upsilon$ & $\upsilon$ & Upsilon \\ \hline
$\Phi$ & $\phi$ & Phi \\ \hline
X & $\chi$ & Chi \\ \hline
$\Phi$ & $\phi$ & Psi \\ \hline
$\Omega$ & $\omega$ & Omega \\ \hline
\end{tabular}
\end{minipage}
\end{table}

\section*{Conjuntos y Listas}
\begin{itemize}
\item Para introducir un conjunto como \{1,2,3\} en Maxima podría utilizar la función \stack{set(1,2,3)}, o utilizar llaves y escribir \stack{\{1,2,3\}}.
\item Las listas pueden introducirse utilizando corchetes. Por ejemplo, para introducir la lista 1,2,2,3 escriba \stack{[1,2,2,3]}.
\end{itemize}

\section*{Ecuaciones e Inecuaciones}
Las ecuaciones se pueden introducir utilizando el signo igual. Por ejemplo, para introducir la ecuación $y=x^2-2x+1$ escriba \stack{y=x$^{\wedge}$2-2*x+1}.

Inecuaciones se pueden introducir como sigue:
\begin{center}
\begin{tabular}{l|l|l|l|l}
Símbolo  & $<$ & $>$ & $\geq$ & $\leq$\\
\hline
Sintáxis & \stack{<} & \stack{>} & \stack{>=} & \stack{<=}
\end{tabular}
\end{center}

Tenga en cuenta que no hay espacio entre estos símbolos y que la igualdad debe ir en segundo lugar cuando se utiliza.

También puede combinar las inecuaciones utilizando operaciones lógicas.  Para introducir el conjunto de números $1\leq x <5$ escriba \stack{1<=x and x<5}.
\end{multicols}

{\tiny \sc Ricardo Largaespada}$~$\hfill{\tiny \today}

\end{document}
